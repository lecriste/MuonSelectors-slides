\documentclass{beamer}
%
% Choose how your presentation looks.
%
% For more themes, color themes and font themes, see:
% http://deic.uab.es/~iblanes/beamer_gallery/index_by_theme.html
%
\mode<presentation>
{
  \usetheme{Warsaw}      % or try Darmstadt, Madrid, Warsaw, ...
  \usecolortheme{default} % or try albatross, beaver, crane, ...
  \usefonttheme{default}  % or try serif, structurebold, ...
  \setbeamertemplate{navigation symbols}{}
  \setbeamertemplate{caption}[numbered]
} 

\usepackage[english]{babel}
\usepackage[utf8x]{inputenc}

\usepackage{pgffor} % \foreach
\usepackage{xstring} % \IfStrEq
\usepackage{ifthen} % \whiledo

\usepackage{extarrows} % \xLongrightarrow

\graphicspath{{pics/ROCs_10_2_X/}}

\title[Muon selectors]{Study of muon selectors performance for low \texorpdfstring{$p_T$}{pT} muons}
\author{Leonardo Cristella}
\institute{University \& INFN of Bari}
\date{Date of Presentation}

\begin{document}

%\tracingall

\begin{frame}
  \titlepage
\end{frame}

% Uncomment these lines for an automatically generated outline.
%\begin{frame}{Outline}
%  \tableofcontents
%\end{frame}

\def \analyses {BPH-15-005, BPH-18-002, BPH-16-004}
\def \titles {{%
"Quarkonium cross sections with early data at c.o.m. energy of 13 TeV",%
"Search for resonances decaying to $\Upsilon(1S)$ + two leptons",%
"Measurement of $B_s \to \mu \mu$ effective lifetime and search for $B^0 \to \mu \mu$"
}}
\def \preSels {{%
"$p_T(\mu) > 3.5 GeV$, $|\eta(\mu)| < 2.4$",%
"$p_T(\mu) > 2.5 GeV$, $|\eta(\mu)| < 1.4$",%
"$p_T(\mu) > 4 GeV$,   $|\eta(\mu)| < 1.4$"
}}

\begin{frame}{Benchmark analyses}
\begin{itemize}
    \foreach \analysis[count=\iAn from 0] in \analyses {
        \item
        \textbf{\analysis} \\
        \pgfmathparse{\titles[\iAn]}
        \pgfmathresult \\
        \pgfmathparse{\preSels[\iAn]}
        \pgfmathresult
        \bigskip
    }
\end{itemize}
\end{frame}

\def \samples {\texorpdfstring{$t\bar{t}$}{ttbar}, \texorpdfstring{$J/\psi \to \mu \mu$}{J/psi -> mumu}, $QCD + Z \to \mu\mu$}
\def \samplesLit {{"ttbar", "JPsiToMuMu", "qcd_zmm"}}

\begin{frame}{Input samples}
\begin{itemize}
    \foreach \sample in \samples {
        \item
        \textbf{\sample}
        \bigskip
    }
\end{itemize}        
\end{frame}

%\begin{block}{Examples}
%Some examples of commonly used commands and features are included, to help you get started.
%\end{block}

\def \matchings {MatchedPrimaryMuon, MatchedMuonFromHeavyFlavour, MatchedMuonFromB, MatchedMuonFromBtoC, MatchedMuonFromC}

\begin{frame}{Signal and background definition}
Among muons passing a given analysis pre-selection cuts
\begin{itemize}
    \item several signal muon definitions studied:
    \begin{itemize}
    \foreach \matching in \matchings {
        \item
        \texttt{\matching}
        %\bigskip
    }
    \end{itemize}        
    \item Background muon ≡ not signal muon
\end{itemize}        
\end{frame}


\foreach \analysis[count=\iAn from 0] in \analyses {
    
    \pgfmathparse{\titles[\iAn]}
    
    \section{\analysis}
    
    \begin{frame}
    \textbf{\analysis} \\
    \pgfmathresult
    \end{frame}

    \foreach \sample[count=\iSamp from 0] in \samples {
        \pgfmathparse{\samplesLit[\iSamp]}
        \IfStrEq{\pgfmathresult}{qcd_zmm}{}{
            \subsection{\sample}
            
            \foreach \matching in \matchings {
                %\whiledo{\equal{\pgfmathresult}{JPsiToMuMu}}{% \and %\equal{\matching}{MatchedMuonFromB}}{ % compilation timeout
                \ifthenelse{\NOT \( \equal{\pgfmathresult}{JPsiToMuMu} \AND \(\equal{\matching}{MatchedPrimaryMuon} \OR \equal{\matching}{MatchedMuonFromHeavyFlavour} \OR \equal{\matching}{MatchedMuonFromC}\) \)}{
                    \def \plot {\analysis /\pgfmathresult_\matching}
                
                    \begin{frame}{\texttt{\matching}}
                    \centering
                    $\xLongrightarrow{\text{log scale}}$
                    
                    \begin{columns}
                        \begin{column}{0.5\textwidth}
                        \includegraphics[width=\textwidth]{\plot.png}
                        \end{column}
        
                        \begin{column}{0.5\textwidth}
                        \includegraphics[width=\textwidth]{\plot-log.png}
                        \end{column}
                    \end{columns}
                    \end{frame}
                    %\breakforeach
                }{}
            } % \foreach \matching
        } % \IfStrEq{\pgfmathresult}{qcd_zmm}{}{
    } % \foreach \sample
} % \foreach \analysis



\begin{frame}{Tables and Figures}

\begin{itemize}
\item Use \texttt{tabular} for basic tables --- see Table~\ref{tab:widgets}, for example.
\item You can upload a figure (JPEG, PNG or PDF) using the files menu. 
\item To include it in your document, use the \texttt{includegraphics} command (see the comment below in the source code).
\end{itemize}

% Commands to include a figure:
%\begin{figure}
%\includegraphics[width=\textwidth]{your-figure's-file-name}
%\caption{\label{fig:your-figure}Caption goes here.}
%\end{figure}

\begin{table}
\centering
\begin{tabular}{l|r}
Item & Quantity \\\hline
Widgets & 42 \\
Gadgets & 13
\end{tabular}
\caption{\label{tab:widgets}An example table.}
\end{table}

\end{frame}

\subsection{Mathematics}

\begin{frame}{Readable Mathematics}

Let $X_1, X_2, \ldots, X_n$ be a sequence of independent and identically distributed random variables with $\text{E}[X_i] = \mu$ and $\text{Var}[X_i] = \sigma^2 < \infty$, and let
$$S_n = \frac{X_1 + X_2 + \cdots + X_n}{n}
      = \frac{1}{n}\sum_{i}^{n} X_i$$
denote their mean. Then as $n$ approaches infinity, the random variables $\sqrt{n}(S_n - \mu)$ converge in distribution to a normal $\mathcal{N}(0, \sigma^2)$.

\end{frame}

\end{document}
